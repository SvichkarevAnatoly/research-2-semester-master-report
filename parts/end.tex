\chapter*{ЗАКЛЮЧЕНИЕ}
\addcontentsline{toc}{section}{ЗАКЛЮЧЕНИЕ}

Целью работы являлось развитие метода ППРЭ,
а именно, разработка механизма асинхронного взаимодействия
метода и минимизируемой функции отклонения
решения математической модели от данных.
За счет единовременной загрузки неизменяемых данных
и асинхронной загрузки переменных параметров
было достигнуто сокращения времени вычисления
при использовании интерпретируемых языков,
таких, как системы статистических расчетов R.

Метод полностью параллельной разностной эволюции
является модификацией стохастического метода оптимизации.
DEEP представляет из себя
эффективный метод решения обратной задачи
математического моделирования,
а именно модификация глобального стохастического метода.

Показана эффективность
оптимизации в сравнении с
прошлой реализацией.
Модификация ППРЭ
была разработана на основе
открытой кроссплатформенной
библиотеки GLib,
что сохраняет кроссплатформенность DEEP.

Использование интерпретируемого языка
для описания математической модели
не является препятствием для применения
эффективного метода оптимизации
и распараллеливания вычислений.
Разработанный механизм обеспечивает
четырёхкратный прирост производительности
на четырёх параллельных потоках
с четырьмя интерпретаторами
по сравнению со старой реализацией.

Работа развивает метод deep
в направлении улучшения интеграции,
что ведёт к увеличению количества
проверяемых гипотез.

