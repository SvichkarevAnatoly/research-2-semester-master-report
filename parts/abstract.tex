\chapter*{РЕФЕРАТ}

% регистрируем счётчики в системе totcounter
\regtotcounter{totalcount@figure}
\regtotcounter{totalcount@table}       % Если поставить в преамбуле то ошибка в числе таблиц
\regtotcounter{TotPages}               % Если поставить в преамбуле то ошибка в числе страниц
\regtotcounter{citenum}

Отчёт \formbytotal{TotPages}{страниц}{а}{ы}{},
1 часть,
\formbytotal{totalcount@figure}{рисун}{ок}{ка}{ков},
\formbytotal{totalcount@table}{таблиц}{а}{ы}{}.
\formbytotal{citenum}{источник}{}{а}{ов}.
\bigskip

Ключевые слова:
МЕТОД ПОЛНОСТЬЮ ПАРАЛЛЕЛЬНОЙ РАЗНОСТНОЙ ЭВОЛЮЦИИ,
ИНТЕРПРЕТАТОР ЯЗЫКА R,
МЕЖПРОЦЕССНОЕ ВЗАИМОДЕЙСТВИЕ,
МАТЕМАТИЧЕСКОЕ МОДЕЛИРОВАНИЕ.

Целью работы является развитие метода ППРЭ,
а именно, разработка механизма
асинхронного взаимодействия метода
и минимизируемой функции.

Учитывая, что для решения задачи
минимизации произвольной целевой функции
не существует универсального алгоритма,
разработка и усовершенствование
методов минимизации остаётся актуальной задачей.

Разработанный механизм позволяет
сократить время вычислений
при использовании интерпретируемых языков,
например, системы статистических расчетов R
за счет единовременной загрузки
неизменяемых данных и
асинхронной загрузки переменных параметров.

Проведено тестирование быстродействия
новой реализации и показана эффективность улучшения.

