\chapter*{ВВЕДЕНИЕ}
\addcontentsline{toc}{section}{ВВЕДЕНИЕ}

Проведение вычислительного эксперимента
с помощью математической модели
в большинстве случаев обходится значительно дешевле,
чем проведение соответствующего эксперимента
над реальным биологическим объектом,
кроме того некоторые условия
невозможно воспроизвести в лаборатории.
Параметры математической модели
должны быть выбраны таким образом,
чтобы получаемые решения
адекватно описывали объект в различных режимах.
Таким образом, разработка методов
и программ для решения обратной задачи
математического моделирования остается актуальной.

В большинстве случаев параметры модели
определяются минимизацией некоторой целевой функции,
описывающей отклонение решений от данных,
с учетом наложенных ограничений.

Методы оптимизации можно классифицировать
в соответствии с задачами оптимизации
на локальные методы,
которые сходятся к локальному экстремуму целевой функции
(в случае унимодальной функции,
экстремум единственный и одновременно
является глобальным экстремумом)
и глобальные методы,
которые стремятся к выявлению
глобальных тенденций поведения целевой функции
и поиску глобального экстремума.

Метод полностью параллельной разностной эволюции
(далее ППРЭ, DEEP) \cite{Kozlov11, Kozlov13}
является модификацией стохастического метода оптимизации,
предложенного в \cite{Storn95}.
DEEP представляет из себя эффективный метод
решения обратной задачи математического моделирования,
а именно модификация глобального стохастического метода.
Это значит, что он способен недетерминированно
(т.е. с использованием вероятностных методов)
находить экстремум для многоэкстремальных целевых функций
только с использованием вычисления целевой функции
в точках приближения без требования вычисления частных производных функции.

Программа DEEP успешно применялась для объяснения
резкого снижения экспрессии в gap генах
на ранних стадиях эмбриогенеза дрозофилы,
причиной которого стала нуль-мутация в гене \textit{Kr} \cite{kozlov2012modeling}.
Недавнее применение DEEP для нахождения
параметров в модели, основанной на
ДНК последовательностях с
регуляторной системой gap генов,
сделало возможным оценивать
вклад каждого сайта связывания
транскрипционного фактора \cite{kozlov2014sequence}.
Остальные применения DEEP
для различных задач описаны в
\cite{kozlov2013enhanced, ivanisenko2014new,
ivanisenko2013replication, kozlov2015differential}.

Целью работы является развитие метода ППРЭ,
а именно, разработка механизма асинхронного
взаимодействия метода и
минимизируемой функции отклонения
решения математической модели от
данных. За счет единовременной загрузки
неизменяемых данных и асинхронной
загрузки переменных параметров планируется
достигнуть сокращения времени вычисления
при использовании интерпретируемых
языков, например, для системы статистических
расчетов R.

Математические модели в биоинформатике
в большинстве случаев создаются
в таких компьютерных системах расчётов
как R, MATLAB, Octave и других.
Нахождение параметров в таких моделях
требует многократного вычисления решений,
что влечёт большие накладные расходы на запуск
того или иного интерпретатора.
Однако, интерпретатор может быть встроен в программу ППРЭ,
что позволит запускать нужное число копий один раз,
и, тем самым сократить время вычислений, для некоторых задач в разы.

Таким образом,
развитие методов решения обратной задачи математического моделирования
и эффективное распараллеливание существующих решений
является важным для исследований системной биологии.

