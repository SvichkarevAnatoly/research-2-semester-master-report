\setcounter{figure}{0} \setcounter{table}{0} \setcounter{equation}{0}
\chapter*{ОСНОВНАЯ ЧАСТЬ}
\addcontentsline{toc}{section}{ОСНОВНАЯ ЧАСТЬ}

\section*{Теория метода DEEP}
\addcontentsline{toc}{subsection}{Теория метода DEEP}

Ниже приведено описание оригинального метода разностной эволюции
и его модификации DEEP.

Разностная эволюция (РЭ) ---
стохастический итерационный алгоритм минимизации,
предложенный Сторном и Прайсом в 1995 г.~\cite{Storn95}.

Метод оперирует случайно сгенерированными векторами параметров,
называемых индивидами. 
Под вектором понимается точка n-мерного пространства
из области определения целевой функции,
которую требуется минимизировать.
Множество индивидов называется популяцией.
Одна итерация популяции РЭ называется поколением.
Первое поколение генерируется случайным образом.
Новое поколение создаётся
по заданной схеме из индивидов текущего поколения.

Идея генерации нового поколения в оригинальном алгоритме
заключается в следующем.
Для каждого индивида текущего поколения
выбираются случайным образом 3 другие индивида
из поколения и вычисляется мутантный вектор по формуле:

\begin{equation} \label{mutant}
    v = v_1 + S \cdot (v_2 - v_3),
\end{equation}

где \begin{math}S\end{math} некоторая положительная константа масштабирования.

Производится операция скрещивания мутантного вектора с исходным,
замещением некоторых координат значениями из исходного вектора.
Полученный вектор называется пробным вектором.
Если значение целевой функции на нём стало меньше,
чем было на исходном, то пробный вектор добавляется в новое поколение.
Если нет, то в новое поколение переходит исходный индивид.
Таким образом, в каждом следующем поколение новые индивиды
стремится уменьшить значение целевой функции
и при определённых условиях может быть найден глобальный минимум.

Опишем две модификации РЭ, реализованные в \cite{KozlovThesis}.

\textbf{Скрещивание с учётом значения функционала:}

Используются два мутантных вектора,
на их основе определяется третий пробный вектор.
Первый мутантный вектор определяется
по соотношению \eqref{mutant},
второй мутантный вектор определяется
по аналогии с правилом треугольника \cite{zaharie2002parameter}:

\begin{IEEEeqnarray*}{rCl}
    z & = & \frac{v_1 + v_2 + v_3}{3}
    + (s_2 - s_1)(v_1 - v_2) \\
    && + (s_3 - s_2)(v_2 - v_3)
    + (s_1 - s_3)(v_3 - v_1), \nonumber
\end{IEEEeqnarray*}

где

\begin{equation*}
    s_i = \frac{\abs{F(q_i)}}
    {\abs{F(q_1)} + \abs{F(q_2)} + \abs{F(q_3)}},
\end{equation*}

для \begin{math}i = 1, 2, 3\end{math}.

Третий пробный вектор составляется из произвольного выбора
соответствующих координат мутантных векторов.
Пробный вектор переходит в новое поколение,
если значение функционала на нём меньше.

\textbf{Скрещивание для поддержания разнообразия индивидов:}

Адаптивная схема для изменения параметров
на основе управления разнообразием в популяции
была предложена в~\cite{fan2003trigonometric}.

Разнообразие популяции можно определить
через дисперсии параметров популяции:

\begin{equation} \label{varj}
    var_j = \frac{1}{NP} \sum_{i = 0}^{NP - 1}
    \left(q_{i, j} - \frac{1}{NP} \sum_{k = 0}^{NP - 1}q_{k, j}\right)^2
\end{equation}
где $j = 0, \dots, I - 1; I$ --- размер вектора параметров.

Введём отдельные константы масштабирования $S_j$ и
вероятности скрещивания $p_j$ для каждого параметра:

\begin{equation*}
    S_j = \left\{ \,
        \begin{IEEEeqnarraybox}[][c]{l?s}
            \IEEEstrut
            \sqrt{\frac{NP \cdot (c_j - 1) + p_j \cdot (2 - p_j)}
            {2 \cdot NP \cdot p_j}} &
            если $NP \cdot (c_j - 1) + p_j \cdot (2 - p_j) \geq 0$, \\
            S_{inf} & иначе.
            \IEEEstrut
        \end{IEEEeqnarraybox}
        \right.
\end{equation*}

\begin{equation*}
    p_j = \left\{ \,
        \begin{IEEEeqnarraybox}[][c]{l?s}
            \IEEEstrut
            -(NP \cdot S_j^2 - 1) +
            \sqrt{(NP \cdot S_j^2 - 1)^2 - NP \cdot (1 - c_j)} &
            если $c_j \geq 1$, \\
            p_{inf} & иначе.
            \IEEEstrut
        \end{IEEEeqnarraybox}
        \right.
\end{equation*}
где
\begin{equation*}
    c_j^{new} = \gamma \cdot \left(\frac{var_j}{var_j^{new}}\right)
\end{equation*}
где $\gamma$ --- новый управляющий параметр,
a $var_j$ определяется по \eqref{varj}.

Управляющие параметры инициализируются
случайным образом, после каждой итерации
обновляются константы масштабирования и
вероятности скрещивания по очереди.

\textbf{Полностью параллельная разностная эволюция:}

Метод РЭ имеет набор управляющих параметров,
которые влияют на скорость работы и сходимость.
К таким параметрам можно отнести
размер популяции, способ рекомбинации,
возраст старейших индивидов.

ППРЭ учитывает возраст индивида
в ходе эволюции.
Возраст индивида ---
число итераций, в течение которых
он не изменяется.
Большой возраст соответствует
попаданию в локальный минимум
функционала качества.
В ППРЭ место старейшего индивида
занимает индивид с наименьшим
значением функционала качества.

ППРЭ представляет собой модификацию метода РЭ,
поддерживающую распараллеливание
процесса вычислений.
Вычисления реализованы в несколько потоков
на многопроцессорных системах.
Каждый индивид помещается в очередь,
из которой извлекается и обрабатывается потоком.
Это позволяет существенно увеличить скорость работы.

